\documentclass[letterpaper]{article} 
\usepackage[left = 0.5in, right = 0.5in, top = 0.9in, bottom = 0.9in]{geometry}
\usepackage{enumitem}
\usepackage{multicol}
\usepackage[spanish]{babel}
\usepackage[utf8]{inputenc}
\usepackage{amsmath,amssymb,amsthm}
\usepackage{tikz-cd}
\usepackage{mathrsfs}
\usepackage[bbgreekl]{mathbbol}
\usepackage{dsfont}
\usepackage{parskip}

\newcommand{\N}{\mathbb N}
\newcommand{\Z}{\mathbb Z}
\newcommand{\Q}{\mathbb{Q}}
\newcommand{\I}{\mathbb{I}}
\newcommand{\R}{\mathbb{R}}
\newcommand{\C}{\mathbb{C}}
\newcommand{\D}{\mathbb{D}}
\newcommand{\F}{\mathcal{F}}
\newcommand{\G}{\mathcal{G}}
\newcommand{\B}{\mathcal{B}}
\renewcommand{\S}{\mathcal{S}}
\newcommand{\E}{\mathbb{E}}
\renewcommand{\P}{\mathbb{P}}
\newcommand{\W}{\dot{W}}
\newcommand{\1}{\mathds{1}}
\newcommand{\abs}[1]{\left\lvert #1 \right\rvert}
\newcommand{\inv}{^{-1}}
\renewcommand{\to}{\rightarrow}
\newcommand{\ent}{\Longrightarrow}
\newcommand{\norm}[1]{\left\Vert #1 \right\Vert}
\renewcommand{\qedsymbol}{$\blacksquare$}


\theoremstyle{definition}
\newtheorem{dfn}{Definición}
\theoremstyle{definition}
\newtheorem{teo}{Teorema}
\theoremstyle{remark}
\newtheorem{proofpart}{Parte}
\theoremstyle{definition}
\newtheorem{cor}{Corolario}
\theoremstyle{definition}
\newtheorem{prop}{Proposición}
\theoremstyle{definition}
\newtheorem{obs}{Observación}
\theoremstyle{definition}
\newtheorem{ejem}{Ejemplo}
\theoremstyle{definition}
\newtheorem{lema}{Lema}


\title{\textbf{Problemas}}
\author{Iván Irving Rosas Domínguez}
\date{\today}

\DeclareSymbolFontAlphabet{\mathbbm}{bbold}
\DeclareSymbolFontAlphabet{\mathbb}{AMSb}
\DeclareMathSymbol\bbDelta  \mathord{bbold}{"01}

\begin{document}
\maketitle
\textbf{Problema 1} 

   En el paper de Maria Emilia se tiene el siguiente Lema como condiciones suficientes para la existencia y unicidad de la densidad continua y acotada para una variable $F$.
   \begin{lema}[Lema 3 en Maria Emilia]
    Sea $F$ una variable aleatoria en el espacio $\D^{1,1}$. Sea $u$ un proceso en $\text{Dom}(\delta)$ tal que cumple las siguientes propiedades, para algunos $p,p'>1$ tales que $1/p+1/p'=1$
    \begin{enumerate}
        \item $u\in L^{p}(\Omega;\mathfrak{H})$
        \item $\delta(u)\in L^{p}(\Omega)$
        \item $(D_uF)^{-1}\in \D^{1,p'}$.
    \end{enumerate}
    Entonces se cumplen las condiciones de la existencia de la densidad continua y acotada del paper de Maria Emilia.
   \end{lema}
   
   Después, se argumenta que las hipótesis del Lema 3 se satisfacen si pedimos las condiciones del siguiente lema.
   \begin{lema}[Lema 4 en Maria Emilia]
    Sean $p,p'>1$ tales que $\frac{1}{p}+\frac{1}{p'}=1$, y sea $F\in \D^{2,1}$. Supongamos que $u\in \text{Dom}(\delta)$. Si se cumplen las siguientes condiciones 
    \begin{enumerate}
      \item $u\in L^{p}(\Omega;\mathfrak{H})$,
      \item $\delta(u)\in L^{p}(\Omega)$,
      \item $(D_uF)^{-1}\in L^{p'}(\Omega)$,
      \item $(D_uF)^{-2}\left(\norm{D^2F}_{\mathfrak{H}^{\otimes 2}}\norm{u}_{\mathfrak{H}}+\norm{Du}_{\mathfrak{H}}\norm{DF}_{\mathfrak{H}}\right)\in L^{p'}(\Omega)$,
    \end{enumerate}
    entonces $\tfrac{u}{D_uF}\in \text{Dom}(\delta)$ y en particular, se cumple la ecuación de la densidad continua y acotada de Maria Emilia.
    \end{lema}
    Básicamente son las mismas hipótesis en ambos lemas, salvo que la condición 3. y 4. del Lema 2 deben implicar la condición 3. del Lema 1. ¿Por qué se da esa implicación?

\textbf{Problema 2.} 

¿Cómo argumentar la cota que aparece en la expresión para la primera derivada de Malliavin de la solución a la ecuación del calor?

\begin{teo}
    Sea $u$ la solución al problema \eqref{shedefinitiva}. Supongamos además que $\sigma:\R\to\R$ es de clase $C^1(\R,\R)$, y su derivada es una función Lipschitz y acotada. Entonces para cualquier $(t,x)\in [0,T]\times\R$, $u(t,x)\in\bigcap_{p\geq1}\D^{1,p}$ y además, la derivada de Malliavin cumple que 
    \[
    D_{s,t}u(t,x)=p_{t-s}(x-y)\sigma(u(s,y))+ \int_{[s,t]}\int_\R p_{t-\tau}(x-\xi)\sigma'(u(\tau,\xi))D_{s,y}u(\tau,\xi)W(d\tau,d\xi),
    \]
    para casi cualquier $s\in [0,t]$ y $y\in \R$. Más aún, se cumple que 
    \begin{equation}
       \norm{D_{s,y}u(t,x)}_{p}\leq C_{T,p \ }p_{t-s}(x-y),
    \end{equation}
    para cualquier $0\leq s<t\leq T$, y para $x,y\in \R$.
  \end{teo}

 \end{document}