\documentclass[letterpaper]{article} 
\usepackage[left = 0.5in, right = 0.5in, top = 0.9in, bottom = 0.9in]{geometry}
\usepackage{enumitem}
\usepackage{multicol}
\usepackage[spanish]{babel}
\usepackage[utf8]{inputenc}

\usepackage{amsmath,amssymb,amsthm}
\usepackage{tikz-cd}
\usepackage{mathrsfs}
\usepackage[bbgreekl]{mathbbol}
\usepackage{dsfont}
\usepackage{parskip}

\newcommand{\op}{\operatorname}
\newcommand{\Op}{^{\op{op}}}
\newcommand{\scc}{\mathscr C}
\newcommand{\scd}{\mathscr D}
\newcommand{\sce}{\mathscr E}
\newcommand{\sci}{\mathscr I}
\newcommand{\scj}{\mathscr J}
\newcommand{\scx}{\mathscr X}
\newcommand{\var}{\mathrm{Var}}
\newcommand{\Id}{\operatorname{Id}}
\newcommand{\N}{\mathbb N}
\newcommand{\Z}{\mathbb Z}
\newcommand{\Q}{\mathbb{Q}}
\newcommand{\I}{\mathbb{I}}
\newcommand{\R}{\mathbb{R}}
\newcommand{\C}{\mathbb{C}}
\newcommand{\F}{\mathcal{F}}
\newcommand{\G}{\mathcal{G}}
\newcommand{\B}{\mathcal{B}}
\newcommand{\abs}[1]{\left\lvert #1 \right\rvert}
\newcommand{\inv}{^{-1}}
\renewcommand{\to}{\rightarrow}
\newcommand{\ent}{\Longrightarrow}
\newcommand{\E}{\mathbb{E}}
\renewcommand{\P}{\mathbb{P}}
\newcommand{\1}{\mathds{1}}
\renewcommand{\qedsymbol}{$\blacksquare$}

\theoremstyle{definition}
\newtheorem{dfn}{Definición}
\theoremstyle{definition}
\newtheorem{teo}{Teorema}
\theoremstyle{definition}
\newtheorem{cor}{Corolario}
\theoremstyle{definition}
\newtheorem{prop}{Proposición}
\theoremstyle{definition}
\newtheorem{obs}{Observación}


\title{\textbf{Convergencia de polinomios y Probabilidad Libre}}
\author{Iván Irving Rosas Domínguez}
\date{\today}

\DeclareSymbolFontAlphabet{\mathbbm}{bbold}
\DeclareSymbolFontAlphabet{\mathbb}{AMSb}
\DeclareMathSymbol\bbDelta  \mathord{bbold}{"01}

\begin{document}
\maketitle

\begin{abstract}
Basado en <<S-transform for finite free probability>>. Trabajo con Fujie, Perales, ?.
ArXiv 2408.09337
\end{abstract}
Trabajo que continúa la tesis de licenciatura de Daniel Perales. Gustó el artículo pero no vieron qué onda hasta recientemente. Conecta dos áreas. Geometría de Polinomios y Matrices aleatorias y probabilidad libre.

El resultado mas importante, y son de esas cosas que cuando uno piensa que unos resultados mágicamente aparecen cuando hace otras cosas, este es un ejemplo. Cosas de matrices aleatorias e ideas de combinatoria se aplican a un problema muy natural que es entender convergencia de polinomios o de su distribución espectral.

El problema que nos interesa es el siguiente.

\begin{enumerate}
    \item[\textbf{1.}] \textbf{Problema}
\end{enumerate}
Sea $(P_n)_{n\geq1}$ sucesión de polinomios de grado $n$, polinomios con raíces reales.
Definimos $P_n(x):=(x-\lambda_1),...,(x-\lambda_n)$. Consideremos la distribución empírica o distribución de raíces $\mu_n:=\sum_{i=1}^n\delta_{\lambda_i}$. Los lambda no necesariamente son distintos (aunque se pueden suponer distintos entre sí)

Nos interesa dar condiciones en los coeficientes para asegurar que $\mu_n\xrightarrow[n\to \infty]{\mathcal{D}}\mu$ y describir de alguna forma la medida $\mu$. 

\textbf{Ejemplos.}
$\cdot$ Polinomios de Hermite. $h_n(x)=\sum_{k=1}^{n}(-1)^k {2n\choose2k} \frac{(2k)!}{k!}\frac{x^{2n-2k}}{(2n)^k}$

Resulta ser que $\mu_{h_n}\to S$, donde se da que Hermite=$\E\left[X_{GUE}\right]$, donde GUE son las matrices Gaussianas Unitarias, y $S$ es la distibución de Wigner.

$\cdot$ Polinomios de Laguerre. $L_n(x)=\sum_{k=0}^{\infty}(x)(-1)^k {n\choose} k\frac{(n)_k}{n^k}X^{n-1}$.

Se tiene que $\mu_{L_n}\to Mp$, donde $Mp$ es la Marchenko-Pasteur y se tiene que $L_n=\E\left[X_{Wishart}\right]$, donde $GG^{T}$, $G=GUE$. (Preguntar a Esaul, en su tesis de maestría estudió este tipo de objetos.)

El problema es determinar a partir de los coeficientes de los polinomios, saber a dónde van. Es decir, ¿qué significa en términos de los polinomios la convergencia de la distribución empírica de las raíces?

$\cdot$ Polinomios de Chebyshev. $Ch(x)=\sum(-1)^{k}\frac{2k!}{k!}\frac{(2n-k-1)!}{2^{2k}}x^{2k}$. 

En este caso $M_{Ch^n}\to arcoseno$. Coincide con $X_{A_G}$, y $A_G$ es un ciclo.

El resultado principal es el siguiente 
\begin{teo}[AFPO]
    Sea $(p_n)_{n\geq1}$ una sucesión de polinomios con raíces simétricas 
    \[
    p_n(x)=x^n-a_1x^{n-1}+a_2x^{n-2}+...
    \] 
    Entonces son equivalentes
    \begin{itemize}
        \item El coeficiente $\frac{{n\choose k-1}a_{k-1}}{{n\choose k}a_k}\to f(t)$ siempre que $\frac{k}{n}\to t$, para cualquier $t$. $k$ depende de $n$, por ejemplo podemos tomar $k=\lfloor nt\rfloor$, para cualquier $t\in [0,1]$.
        \item $\mu[P_n]\xrightarrow[n\to \infty]{\mathcal{D}}\mu$.
    \end{itemize}
    Observamos que como función los $P_n$ no convergen, esa es la dificultad. Nos ponemos a estudiar las medidas empíricas. Mimbela pregunta. ¿Pero y qué pasa con polinomios que sí convergen? Por ejemplo, los polinomios de Bernstein. ¿Da algo de intuición? ¿Se aplica este resultado para esos polinomios?
 \end{teo}

 En el ejemplo de los polinomios de Laguerre, resulta que 
 \[
 \frac{(n)_k}{n^{k}}/\frac{(n)_{k+1}}{n^{k+1}}=\frac{n}{n-k+1}.
 \]
 Por lo tanto, si $k\approx tn$, entonces lo anterior converge a $\frac{1}{1-t}=f(t)$. Y esta madre es creciente.

 Lo mismo se puede hacer en Cheby o en otros. Ahora bien, en el caso de Laguerre, la $f$ resultó ser relativamente sencilla. En el problema general ¿quién es la $f$?

 \begin{enumerate}
    \item[\textbf{2.}] \textbf{Elementos de Proba Libre.}
 \end{enumerate}
 Recordemos que la transformada de Cauchy es
 \[
 G_\mu(z)=\int_{-\infty}^\infty\frac{1}{2-t}d\mu
 \]
con $z\in \C$, y además uno puede recuperar la densidad de la medida como 
\[
\lim_{y\to 0^+}Im(G_\mu (x+iy))=\pi d\mu.
\]
Ahora, nótese que 
\[
\psi_{\mu}(z)=\int_{0}^{\infty}\frac{tz}{1-tz}d\mu(t)=\frac{1}{z}G_\mu(\frac{1}{z})-1
\]
y definimos la transformada $S$ como 
\[
S_\mu(z)=\frac{1+z}{z}\psi^{-1}_\mu(z)
\]
Nos preguntamos ¿cuáles son las propiedades de la transformada $S$? 

\textbf{Propiedades de la transformada $S$}.
\begin{itemize}
    \item $S_\mu:(-1,0)\to\R^{+}$, con $\lim_{t\to-1}S_\mu(t)=a$, con $a=\int x^{-1}d\mu$ y también  $\lim_{t\to 0}S_\mu(t)=b^{-1}$ donde $b=\int x d\mu$.
    \item $S_\mu$ es estrictamente creciente.
    \item $S_{\mu\otimes\nu}=S_\mu S_\nu$.
\end{itemize}
¿Cuál fue el resultado entonces? Resulta ser que, por ejemplo, la transformada $S_\mu$ de la Marchenko Pasteur resulta ser $f(t)=\frac{1}{1-t}$. Entonces dada la equivalencia que se tiene mostrada en el teorema principal, podemos calcular el límite de la primera condición (si existe el límite, existe la transformada $S$ de la medida empírica). Luego, calculamos transformadas $S$ de distribuciones, analizamos a donde converge la medida empírica, le sacamos la transformada $S$ a la medida límite y tratamos de cuadrar. Vamos, que en resumen, el teorema que se probó se puede reescribir como 

\begin{teo}[AFPO]
    Sea $(p_n)_{n\geq1}$ una sucesión de polinomios con raíces simétricas 
    \[
    p_n(x)=x^n-a_1x^{n-1}+a_2x^{n-2}+...
    \] 
    Entonces son equivalentes
    \begin{itemize}
        \item El coeficiente $\frac{{n\choose k-1}a_{k-1}}{{n\choose k}a_k}\to S_\mu(-t)$. 
        \item $\mu[P_n]\xrightarrow[n\to \infty]{\mathcal{D}}\mu$.
    \end{itemize}
 \end{teo}
 Lo anterior bajo algunas hipótesis.
\begin{enumerate}
    \item[\textbf{3.}] \textbf{Polinomios} 
\end{enumerate}
Nótese que 
\[
P_n(x)=(x-\lambda_1)...(x_{\lambda_n})=x^n-e_1x^{n-1}+...
\]
donde $e_i=\sum_{1\leq i_1,...,i_k\leq n}\lambda_{i_1}...\lambda_{i_k}$.
y donde $\tilde{e}_i:=e_k/{n\choose k}$. Además estos últimos satisfacen propiedades de convexidad:
\[
\tilde{e}_{k+1}\tilde{e}_{k-1}\geq\tilde{e}_{k}^2.
\]
y se satisface que 
\[
\tilde{e}_n/\tilde{e}_{n-1}\leq \tilde{e}_{k}/\tilde{e}_{k+1}\leq\tilde{e}_{k-1}/\tilde{e}_{k}\leq...\leq\tilde{e}_0/\tilde{e}e_{1}.
\]
Y entonces 
\[
\frac{\lambda_1\cdot...\cdot\lambda_n}{\lambda_1\cdot...\cdot\lambda_{n-1}+...}=\frac{1}{1/\lambda_1+...+1/\lambda_{n}}
\]
A Daniel Perales se le ocurrió que lo anterior fuera la transformada $S$ para polinomios. Se comporta exactamente de la misma manera que la transformada $S$.

\begin{teo}[Hoskens-Kabluchko, Ar. Perales-Garza]
 $\mu_{P_n}\to \mu$ y si $k\sim tn$, entonces $\mu_{P_n}^(k)\to D_t\mu^{\boxplus 1/t}=\mu_t$.
 Y se tiene que 
 \[
 S_{\mu_t}(tz)=S_\mu(z).
 \]
 \end{teo}
Veamos. Si tenemos el polinomio, ¿cómo son los coeficientes de las derivadas? 
\[
\partial_x P_n(x)=nx^{n-1}-(n-1e_1x^{n-2}).
\]
Lo vuelvo mónico, y nos damos cuenta del lema siguiente
\begin{teo}[Lema]
    \[
        \tilde{e}_k(P)/\tilde{e}_{k-1}(P)=\tilde{e}(\partial_xP)/\tilde{e}_{k-1}(P)\cdot...\cdot\tilde{e}_k(\partial^lP)/\tilde{e}_{k-1}(P).
    \]
    Observamos que tomando convergencias, se tienen las siguientes relaciones:

    \[
        \tilde{e}(\partial_xP)/\tilde{e}_{k-1}(P)\cdot...\cdot\tilde{e}_k(\partial^lP)/\tilde{e}_{k-1}(P)\to \int x^{-1}\partial^k(P_k)\to\int x^{-1}\mu_t=S_{\mu_t}(-1)=S_\mu(-t).
    \]
 \end{teo}
 \end{document}